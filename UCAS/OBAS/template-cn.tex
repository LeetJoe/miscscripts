%!TEX program = xelatex
\documentclass[conference]{IEEEtran}
\IEEEoverridecommandlockouts

\usepackage[UTF8]{ctex}
\usepackage{fancyhdr}
\usepackage{cite}
\usepackage{epsfig}
\usepackage{xcolor}
\usepackage{booktabs}
\usepackage{array}
\usepackage{multirow}
\usepackage{textcomp}
\usepackage{tabularx}
\usepackage{verbatim}
\usepackage{graphics,graphicx}
\usepackage{algorithm,algpseudocode}
\usepackage{amsmath,amssymb,amsfonts}
\usepackage{bm}
\usepackage{cases}
\usepackage[numbers,sort&compress]{natbib}
\usepackage{amsmath,empheq,eqparbox}
\usepackage[perpage,symbol,flushmargin]{footmisc}

\renewcommand*\footnoterule{\hrule width 2.5cm height 0.4pt \vspace*{0.5ex}}

\newcommand{\eqmath}[3][c]{%
	% #1 = alignment, default c, #2 = label, #2 = math material
	\eqmakebox[#2][#1]{$\displaystyle#3$}%
}
\newcommand{\eqtext}[3][c]{%
	% #1 = alignment, default c, #2 = label, #2 = text material
	\eqmakebox[#2][#1]{#3}%
}

\renewcommand{\algorithmicrequire}{\textbf{Input:}}
\renewcommand{\algorithmicensure}{\textbf{Output:}}

\def\BibTeX{{\rm B\kern-.05em{\sc i\kern-.025em b}\kern-.08em
    T\kern-.1667em\lower.7ex\hbox{E}\kern-.125emX}}

\fancypagestyle{mystyle}{%
	\fancyhf{} % 清空页眉和页脚的设置
	\fancyhead[LE,RO]{\thepage} % 在偶数页左侧和奇数页右侧显示页码
	\fancyhead[RE]{\leftmark} % 在偶数页右侧显示当前章的标题
	\fancyhead[LO]{\rightmark} % 在奇数页左侧显示当前节的标题
	\renewcommand{\headrulewidth}{0.4pt} % 设置页眉下方横线的粗细为0.4pt
	\renewcommand{\footrulewidth}{0pt} % 页脚上方无横线
}

\begin{document}
	
\pagestyle{mystyle}

\title{《如何成为一名科学家》读书报告}

\author{
    \IEEEauthorblockN{
       宋超$^{\dagger, \ddagger}~$,202328020629002
    }
    \IEEEauthorblockA{
       songchao2023@ia.ac.cn \\
       \textit{$^{\dagger}$中国科学院自动化研究所}\\
       \textit{$^{\ddagger}$中国科学院大学}
    }
}

\maketitle

\begin{abstract}
《如何成为一名科学家(On Being A Scientist)》\cite{a1989being}是由(美国)科学、工程与公共政策委员会、(美国)国家科学院、(美国)国家工程院、(美国)医学研究所共同撰写的关于科研人员在日常研究中应该如何负责任地开展工作的指导性书籍。

它为科学研究行业提供了一份职业规范纲要,并论证了在科研工作中遵守这些规范的重要性,对从研究生到博士后再到职业科研人员各阶段都有积极的指导意义。

书中就科研人员应该如何在日常研究和论文发表中规范自己的行为、如何处理好自身利益与其它相关各方利益的关系,分列了科研工作中有代表性的十二个主题,结合具体案例进行了深入的分析和讨论,主题涵盖了从师生关系到署名权、利益冲突、学术不端等,对处于研究生阶段、有志从事科研工作的学生非常具有启发意义。
\end{abstract}


\section{师生关系}

良好的师生关系不仅对初入研究领域的学生产生非常大的作用,导师自己也可以在合作网络以及声誉和尊重方面获益。然而在相处过程中,利益分配、文章发表、权责划分等方面也很容易出现或大或小的冲突。
导师应该意识到自己对学生的学术生涯所能产生的重要影响,所以要非常小心地避免权威的滥用,形成良好的行为榜样,对同行也会形成示范作用。同样,学生应该对自己导师负责,跟导师明确自己可以工作和面谈的时间。学生还有责任主动去寻找能够给自己提供指导的老师,而不能指望会有合适的导师主动找上自己。
对于研究所而言,很难明确规定在良好师生关系中,师生具体应该如何做;不过研究所采取各种方式,鼓励和表彰具有良好品行的导师,将会产生非常积极的作用。


\section{数据处理}

在实际操作中,对数据进行不适当的人为处理的现象并不少见,信息处理和传输技术的发展使得这一现象更普遍且难于鉴别。不同行业里对数据的处理会有不同的要求,但是创建并维护准确、可用并永久存档的数据以备他人对论文进行验证和复现应该是一项基本的义务,新手研究者尤其要注意此问题。很多研究机构都对数据的披露有明文要求,但是实验材料稀缺、不易保存或者实验数据的保存介质问题会给实际操作造成很多困难,机构和研究者应该相互协作,采用一种可行的方式。


\section{失误与怠惰}

科学研究对错误是非常敏感的,然而完全避免错误是非常困难的。一方面,在某些新的领域,没有成熟的观测方法,在尝试过程中难免引入各种误差;再者,无论如何小心,人总是无法避免失误或怠惰,由此引入的错误也不可能完全杜绝。
即便如此,不论是出于对公众、对所在专业,还是对自己负责的目的,研究者都有义务谨慎地避免此类错误。研究结果必须经过充分的准备并提交同行评议,哪怕在发表之后,也要持续跟踪审视。
任何错误都可能对他人产生误导,甚至造成巨大的浪费和损失并损害个人声誉,因此研究者不能寄希望于通过同行评议或者公众监督来发现错误;在文章发表之后如果发现了错误,要及时更新文章,制订勘误表等。

\section{科研不端}

科研不端行为是指那些在科研工作中严重违背科研核心准则因而被学会和研究机构严厉对待的行为,可以说是从事科学研究工作的红线,一旦触及很可能断送一个人的科研生涯。
由美国科技政策办公室发布并被广泛认可的关于科研不端行为定义为:在科研工作的开题、执行或审查工作中,或者在报告科研成果时所发生的伪造(fabrication)、变造(falsification)与剽窃(plagiarism)行为,简称为“FFP”。有一些科研机构或者科研基金代理会对科研不端行为的定义进行扩展,但是无论定义如何,其与一般的错误或者疏忽的关键差别就在于是否“有意欺骗”。明知是不端行为,仍然违反相关规定是对科学基石严重破坏。
科研不端行为的后果是非常严重的,比如大量时间的浪费、声誉的破坏以及个人情感的伤害,甚至对科研行业乃至整个社会造成不良影响,每一个科研工作者都必须谨言慎行,严防此类行为的出现。

\section{对疑似违反专业规范行为的处理}

科学研究行业基本是自我规束的,虽然政府通常也会出台一些相关的法规,研究委员会也会出台一些希望所有科研人员遵守的行为准则与规范。自我规束保证了关于专业操守的决定都是由有足够经验和资历的同行完成的,但要确保自我规束的正常运行,科研人员在发觉自己的同行存在有违科研规范和行为准则有必要发出提醒。然而自我规束也会面临很多问题,比如匿名性的保证、对举报者的报复行为的禁止、对恶意诬告的处理等,而且同行监督的义务并非强制性的,科研不端行为还会破坏这种自我规束机制的有效运行。
对于可疑行为,提出“质疑”要比直接“指控”更好。科学家和研究机构可以依据各自不同的情况,定义一些“有问题的研究实践(questionable research practices, QRPs)”。如果对某些行为或情况心有疑虑,寻求值得信任的朋友或者导师的意见也是一种值得尝试的途径,必要的时候可以组织开展讨论。在决定要举报一些可疑行为时,要注意遵守相关的指导方针,并对自己举报的动机、是否怀有偏见等进行自我反思。

\section{其它主题}

另外,本书还就“科学研究中的人类参与者和动物实验对象”、“科学研究中的实验室安全”、“研究结果的发表”、“署名与信誉分配”、“知识产权”、“利益、承诺与价值上的冲突”以及“社会环境中的研究者”等主题进行了展开讨论。

当科学研究涉及人类参与者和作为实验对象的动物时,应充分尊重和保护受试者的权益,参考《实验室动物关照与使用指南》执行操作;在实验室安全方面,要注意科学研究中接触到的实验设备或危化物品等的使用,严格遵守实验室规定;在发表研究结果时,要认真接受或者参与同行评议,避免一稿多搞、过度引用等问题;论文署名时应知会所有参与,并按对论文的实际贡献大小来排名,不适合署名的贡献者应写在致谢部分中;申请知识产权的时候应遵守实验室和研究所的相关规定并充分尊重它们的权益;利益冲突也是一个比较常见且敏感的问题,这类问题比较复杂,需要遇到此问题的科研工作者谨慎面对、主动避嫌,以免被诟病或者卷入不必要的纠纷;研究者也是社会中的一员,应该仔细对待自己作为科研工作者与整个社会的关系,牢记自己的品行会影响社会对科研行业、对科学家的态度的影响,严格要求自己才有利于科研行业和整个社会协同发展与进步。

\section{结语}

本书虽然简要但是全面地介绍了科学研究工作中广泛存在的各类问题,并结合实例对所有问题进行了细致的分析与讨论。通过学习本书的内容,可以让尚处于研究生阶段的准科研工作者提前了解各方面的问题,对其后续接触、开展科研工作可以起到很好的指导作用。


\bibliographystyle{IEEETran}
\bibliography{sctest}

\end{document}














