%!TEX program = xelatex
\documentclass[conference]{IEEEtran}
\IEEEoverridecommandlockouts

\usepackage[UTF8]{ctex}
\usepackage{cite}
\usepackage{epsfig}
\usepackage{xcolor}
\usepackage{booktabs}
\usepackage{multirow}
\usepackage{textcomp}
\usepackage{tabularx}
\usepackage{verbatim}
\usepackage{graphics,graphicx}
\usepackage{algorithm,algpseudocode}
\usepackage{amsmath,amssymb,amsfonts}
\usepackage[numbers,sort&compress]{natbib}

\renewcommand{\algorithmicrequire}{\textbf{Input:}}
\renewcommand{\algorithmicensure}{\textbf{Output:}}

\def\BibTeX{{\rm B\kern-.05em{\sc i\kern-.025em b}\kern-.08em
    T\kern-.1667em\lower.7ex\hbox{E}\kern-.125emX}}

\begin{document}

\title{论文题目}

\author{
    \IEEEauthorblockN{
       宋超$^{\dagger, \ddagger}~$,202328020629002
    }
    \IEEEauthorblockA{
       电子邮箱地址,如 songchao2023@ia.ac.cn \\
       \textit{$^{\dagger}$中国科学院自动化研究所}\\
       \textit{$^{\ddagger}$中国科学院大学}
    }
}

\maketitle

\begin{abstract}
  给出本篇论文简短的摘要。
\end{abstract}


\section*{课程论文要求}

论文需与认知计算课程内容相关,可以提出或改进算法,可以使用算法解决专业领域问题,也可以是认知计算或人工智能相关领域文献综述(近三年)。

要求:
\begin{enumerate}
  \item 论文要有实验数据支撑(文献综述可以不包含实验,但是要有充足的文献支撑),需要引用相关的参考文献,参考文献请统一使用本模板包含的 IEEETran 格式;
  \item 中文或英文撰写,正文不少于 6 页(不含参考文献),课程网站已上传中文 Word 模板、中英文 \LaTeX 模板,任选其一使用;
  \item 不得抄袭,不得直接翻译已发表文章,不得与往届论文雷同。一旦发现上述任一情形,本门课程不通过。
\end{enumerate}

注意事项:
\begin{enumerate}
  \item 介绍一个工作前,最好花一定篇幅讲清楚问题定义;
  \item 自己提出的方法或调研文献中的方法,除描述方法和实验结果外,可以着重分析方法的优缺点(按照自己的理解),并针对缺点尝试给出可能的改进方案;
  \item 仔细检查交叉引用是否全部编译成功,行文中避免出现 ``?'',尽量减少错别字和语法错误;
  \item 参考文献中必不可少的部分:作者、题目、发表期刊或会议、发表时间。
\end{enumerate}

加分项:
\begin{enumerate}
  \item 提出的算法有创新性;
  \item 解决的问题有实际或科研价值;
  \item 文献综述追踪前沿进展,分析总结独到全面。
\end{enumerate}

以下章节仅供参考,可根据内容调整。

\section{引言}

中文论文请全文使用中文标点,除专业术语外不要出现英文;英文论文请全文使用英文标点,不要出现任何中文。

\section{相关工作}

参考文献引用方法,如图卷积神经网络\cite{kipf2017semi},图注意力网络\cite{velivckovic2018graph}等。

\section{模型}

最好有配图和公式。

\section{实验}

实验数据支撑。

\section{结论}

论文要有明确的结论。

\bibliography{reference_list}
\bibliographystyle{IEEETran}

\end{document}
