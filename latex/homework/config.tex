\usepackage[a4paper,top=2.54cm,bottom=2.54cm,left=2.7cm,right=2.7cm]{geometry}

% 载入常用的数学包, 符号包
\usepackage{amsmath}
\usepackage{mathtools}
\usepackage{amsfonts}
\usepackage{amssymb}
\usepackage{mathrsfs}
\usepackage[justification=centering]{caption}
\usepackage{booktabs}
\usepackage{blindtext}
\usepackage{calc} % 用于使用宽度和长度的计算

%% linespace 行间距,段间距等等
\usepackage{setspace}

\setlength{\parindent}{2em}
\fontsize{12pt}{15pt}\selectfont    % 小四号,1.25倍行距
% fonts (style, color, size).
\usepackage{ctex}		 	% If you are lazy, the CTEX suit is enough.

% Chinese font
\usepackage{xeCJK}		 	% For the Chinese through XeLaTex
\setCJKmainfont{SimHei.ttf} 	% set the mainfont of Chinese as songti. (serif) for
\setCJKsansfont{SimHei.ttf}	% sans serif font for \textsf
\setCJKmonofont{SimHei.ttf}	% monospace font for \texttt
% \punctstyle{kaiming}   	% Remove the space used by symbols like comma.
\setCJKfamilyfont{song}{SimSun.ttf}
\newcommand{\song}{\CJKfamily{song}} %宋体 song
\setCJKfamilyfont{kai}{SimKai.ttf}
\newcommand{\kai}{\CJKfamily{kai}} %楷体2312  kai
\setCJKfamilyfont{hei}{SimHei.ttf}
\newcommand{\hei}{\CJKfamily{hei}} %黑体  hei

% English font
\usepackage{fontspec}
\setmainfont{Times New Roman}
\setsansfont{Times New Roman}
\setmonofont{Times New Roman}
% font Size
\newcommand{\yihao}{\fontsize{26pt}{18pt}\selectfont}
\newcommand{\erhao}{\fontsize{22pt}{18pt}\selectfont}
\newcommand{\xiaoerhao}{\fontsize{18pt}{18pt}\selectfont}
\newcommand{\sanhao}{\fontsize{16pt}{18pt}\selectfont}
\newcommand{\sihao}{\fontsize{14pt}{17.5pt}\selectfont}
\newcommand{\xiaosihao}{\fontsize{12pt}{15pt}\selectfont}
\newcommand{\wuhao}{\fontsize{10.5pt}{13.125pt}\selectfont}


% set the styles of sections at all levels
\usepackage{titlesec}
\usepackage{titletoc}
\titleformat{\section}{\hei\bfseries\sihao}{\thesection.}{1em}{} % 在section标题编号后面加个点
\titleformat{\subsection}{\raggedright\hei\bfseries\xiaosihao}{\thesubsection.}{1em}{}
\titleformat*{\subsubsection}{\raggedright\hei\bfseries\xiaosihao}
\titleformat{\paragraph}[hang]{\raggedright\hei\bfseries\xiaosihao}{\theparagraph}{1em}{}[]

\usepackage{abstract} % 引入abstract宏包

% 设置摘要的格式
\renewenvironment{abstract}%
  {% 开始
    \begin{center}%
    \bfseries % 加粗
    \erhao % 字体大小
    \thetitle % 显示"摘要"标题
    \end{center}%
    \par
    \xiaosihao
    \noindent
    \textbf{摘要:}
  }%
  {% 结束
    \par
    \vspace{1em}
    \noindent
    \textbf{关键词:} % 显示"关键词:"标题
  }
\newcommand{\keywords}[1]{\xiaosihao #1}


% reference and citation 参考文献
\usepackage[numbers]{natbib}
\renewcommand{\refname}{\hei\sihao \centerline{参考文献}\vspace{-1em}}
\bibsep=0pt % 用来设置每个\bibitem之间的间距
% \newcommand{\upcite}[1]{\textsuperscript{\textsuperscript{\cite{#1}}}} % show citation label in the upperscript

\usepackage{listings} % For the code. 代码
\usepackage[bookmarks=true,colorlinks,linkcolor=black,citecolor=black,urlcolor=purple]{hyperref}
\usepackage{appendix}
\renewcommand{\appendixname}{附录}

% titlepage
\usepackage{titling}
% 重置命令 maketitle
\renewcommand{\maketitle}{
	\def\titlelength{8em}
 	\begin{titlepage}
		\begin{center}
			\vspace*{2em}
            {\sihao \hei 中国科学院大学人工智能学院《文献阅读》课程—文献综述报告}\\
            \vspace*{-0.5em}
            \rule{\textwidth}{0.5pt}\\
			\vspace*{3em}
			{\sihao \hei \bfseries 课程编号:}
            {\sihao \courseID}\\
			\vspace*{11em}
			{\yihao \hei \bfseries \thetitle}

			\vspace*{9em}
			{\sanhao \hei
				\renewcommand\arraystretch{1.5}
				\begin{tabular}{ll}
					\makebox[4em][r]{{\bfseries 撰写人:}} &
					\makebox[\titlelength][l]{\theauthor} \\
					\makebox[4em][r]{{\bfseries 学号:}} &
					\makebox[\titlelength][l]{\studentID} \\
					\makebox[4em][r]{{\bfseries 培养单位:}} &
					\makebox[\titlelength][l]{\organization} \\
			  \end{tabular}
		    }

			\vspace{4em}
			{\sanhao \bfseries \thedate}
		\end{center}
	\end{titlepage}
}

