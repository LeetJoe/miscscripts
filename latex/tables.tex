\documentclass{article}

% if you need to pass options to natbib, use, e.g.:
%     \PassOptionsToPackage{numbers, compress}{natbib}
% before loading neurips_2019

\PassOptionsToPackage{numbers}{natbib}

% ready for submission

% to compile a preprint version, e.g., for submission to arXiv, add add the
% [preprint] option:
%     \usepackage[preprint]{neurips_2019}

% to compile a camera-ready version, add the [final] option, e.g.:
%     \usepackage[final]{neurips_2019}

% to avoid loading the natbib package, add option nonatbib:
%     \usepackage[nonatbib]{neurips_2019}

\usepackage[utf8]{inputenc} % allow utf-8 input
\usepackage[T1]{fontenc}    % use 8-bit T1 fonts
\usepackage{booktabs}       % professional-quality tables
\usepackage{microtype}      % microtypography

\usepackage{subfigure}
\usepackage{multirow}
\usepackage{boldline}
\usepackage{cellspace}
\usepackage{color,verbatim}
\usepackage[tableposition=top]{caption}
\usepackage{hyperref}
\usepackage{enumitem}
%\usepackage{graphics}
\usepackage{amssymb}
\usepackage{floatrow}
\usepackage{colortbl}

%\usepackage{subfig}
\usepackage{graphicx}
\usepackage{floatrow}

\usepackage{xcolor}

\usepackage{floatrow}
\newfloatcommand{capbtabbox}{table}[][\FBwidth]

\usepackage{blindtext}
\floatsetup[table]{capposition=top}


\usepackage{array}
\newcolumntype{L}[1]{>{\raggedright\let\newline\\\arraybackslash\hspace{0pt}}m{#1}}
\newcolumntype{C}[1]{>{\centering\let\newline\\\arraybackslash\hspace{0pt}}m{#1}}
\newcolumntype{R}[1]{>{\raggedleft\let\newline\\\arraybackslash\hspace{0pt}}m{#1}}


\title{Beautifule Tables}

\begin{document}
	
	\maketitle


	

	\begin{table}[htbp]
		\centering
		\caption{MP-Net 在 icews14 上的表现}
		\begin{tabular}{cccc}
			\toprule  % 顶部线
			MMR&Hit@1&Hit@3&Hit@10 \\ 
			\midrule  % 中部线
			0.7170&0.6731&0.7360&0.8030 \\
			\bottomrule  % 底部线
		\end{tabular}
	\end{table}

	
	\begin{table}[bht]
		\caption{Comparison of result in the paper and what we get. H@K is in \%.}
		\label{tab::results-kge}
		\begin{center}
			\scalebox{0.75}
			{
				\begin{tabular}{c c |c c c c c | c c c c c}\hline
					\multirow{2}{*}{\textbf{KGE Method}}	& \multirow{2}{*}{\textbf{}} & \multicolumn{5}{c|}{\textbf{FB15k}} & \multicolumn{5}{c}{\textbf{WN18RR}}	\\
					& & \textbf{MR} & \textbf{MRR} & \textbf{H@1} & \textbf{H@3} & \textbf{H@10} & \textbf{MR} & \textbf{MRR} & \textbf{H@1} & \textbf{H@3} & \textbf{H@10} \\ 
					\hline
					& in paper & 33 & 0.792 & 71.4 & 85.7 & 90.1 & 3436 & 0.230 & 1.5 & 41.1 & 53.1 \\
					\rowcolor{green!60}  % 结合下一行看,如果把行合并写在上面,对不合并的行设置 rowcolor 
													% 会把合并行的内容挡住;写成下面的形式,合并行的内容结构上归属
													% 下面这行,对本行设置 rowcolor 就不会遮挡其内容。
					\multirow{-2}{*}{\textbf{TransE}}  % 这里的负号表示向上合并两行,对列也有类似处理
					& we get & 36 & 0.792 & 74.5 & 82.2 & 87.5 & 3481 & 0.438 & 39.6 & 44.0 & 53.2 \\
					\hline
					& in paper & 40 & 0.791 & 73.1 & 83.2 & 89.5 & 4902 & 0.442 & 39.8 & 45.5 & 53.5 \\
					\rowcolor{green!20}
					\multirow{-2}{*}{\textbf{DistMult}} 
					& we get & 36 & 0.743 & 67.7 & 78.1 & 86.4 & 4612 & 0.443 & 40.1 & 45.4 & 53.5 \\
					\hline
					& in paper & 39 & 0.776 & 70.6 & 81.7 & 88.5 & 5266 & 0.471 & 43.0 & 49.2 & 55.7 \\
					\rowcolor{green}
					\multirow{-2}{*}{\textbf{ComplEx}} 
					& we get & 40 & 0.765 & 70.7 & 79.9 & 87.2 & 5214 & 0.468 & 42.8 & 48.2 & 54.8 \\
					\hline
				\end{tabular}
			}
		\end{center}
		\vspace{-0.15cm}
	\end{table}

\end{document}
